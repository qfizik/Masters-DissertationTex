\documentclass[../../dissertation.tex]{subfiles}
\begin{document}

A computação quântica é uma área emergente, que junta os campos de Mecânica
Quântica, Ciências da Computação e Teoria da Informação, com a promessa de
melhoramentos a algoritmos clássicos tais como a simulação de sistemas
quânticos, criptografia, busca em base de dados, e outros. Entre estes
algoritmos, as caminhadas quânticas surgem com um ganho quadrático de
complexidade em comparação às caminhadas clássicas, possibilitando melhor
desempenho em aplicações como distinção de elementos, problemas de busca,
verificação de produtos de matrizes e tempos de alcance em grafos. O trabalho atual
oferece uma visão geral de um ponto de vista teórico, de simulação e de
implementação de circuitos, relativos aos modelos de caminhadas quânticas com
moeda, escalonadas e continuas no tempo. Os primeiros dois capítulos desta tese
são dedicados à definição da estrutura teórica, simulação em Python e
comparação dos modelos supracitados, para o caso simples da
dinâmica na linha, e para o problema de busca num grafo completo. Isto será
então utilizado como referência para o capitulo final, onde as principais
contribuições da tese acontecem. Os circuitos para a dinâmica e
problema de busca são apresentados, de modo a correr estes algoritmos nos
computadores supercondutores da IBM. 

\end{document}

\documentclass[../../dissertation.tex]{subfiles}
\begin{document}

A computacao quantica e uma area emergente que junta os campos de mecanica
quantica, ciencias da computacao e teoria de informacao, com a promessa de
melhoramentos a algoritmos classicos tais como a simulacao de sistemas
quanticos, criptografia, busca em base de dados, entre outros. Entre estes
algoritmos, as caminhadas quanticas surgem com um ganho quadratico de
complexidade em comparacao as caminhadas classicas, possibilitando assim melhor
desempenho em aplicacoes como distincao de elementos, problemas de busca,
verificacao de produtos de matrizes e hitting times em grafos. O trabalho atual
oferece uma visao geral de um ponto de vista teorico, de simulacao e de
implementacao de circuitos relativos aos modelos de caminhadas quanticas com
moeda, escalonadas e continuas no tempo. Os primeiros dois capitulos desta tese
sao dedicados a definicao da estrutura teorica, simulacao em Python e
comparacao das caminhadas quanticas supracitadas, para o caso simples da
dinamica na linha e para o problema de busca num grafo completo. Isto sera
entao utilizado como referencia para o capitulo final, onde as principais
contribuicoes da tese acontecem. Neste capitulo, os circuitos para a dinamica e
problema de busca sao apresentados, de modo a correr estes algoritmos nos
computadores supercondutores da IBM. 

\end{document}

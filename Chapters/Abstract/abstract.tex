\documentclass[../../dissertation.tex]{subfiles}
\begin{document}

Quantum computing is an emergent field that brings together quantum mechanics,
computer science and information theory, which promises improvements to
classical algorithms such as simulation of quantum systems, cryptography, data
base searching and many others. Quantum computing offers several different
models for quantum walks, which have complexity advantages over their classical
counterparts and many applications among which are element distinctness,
searching problems, matrix product verification and hitting times in graphs.
The present work offers a general theoretical overview, simulation and circuit
implementation of the coined, staggered and continuous-time quantum walk
models. The first two chapters of this thesis define the theoretical framework,
simulate in Python and compare the aforementioned quantum walk modelssimple for
the simple case of the dynamics in a line graph and for the search problem case
in a complete graph.  The final chapter contains the main contributions of this
thesis, where the circuit implementations for the dynamics and serch problem
are created in order to run these algorithms in IBM's superconducting NISQ
computers.
%TODO: Mencionar que fazemos comparacao entre os modelos, vantagens e
%desvantagens.
%TODO: Analise do ponto de vista teorico, do ponto de vista de simular que funciona como benchmark, analise de circuito de cada um e implementacao num computador NISQ de supercondutor.

\end{document}

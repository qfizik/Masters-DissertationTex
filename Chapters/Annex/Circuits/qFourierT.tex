\documentclass[../../../dissertation.tex]{subfiles}
%\DeclareMathOperator*{\Motimes}{\text{\raisebox{0.25ex}{\scalebox{0.8}{$\bigotimes$}}}}
\begin{document}
As was seen in section \ref{sec:chapGrover}, quantum computers can perform certain tasks more efficiently than classical computers. Another such example is the problem of finding the prime factorization of an $n$-bit integer, which the most efficient solution to date, proposed by \cite{Pollard93}, requires $e^{O(n^{\frac{1}{3}}\log^{\frac{2}{3}}n)}$ operations. In contrast, a quantum algorithm proposed by \cite{Shor94} accomplishes the same task in $O((\log n)^2 (\log \log n) (\log \log \log n))$ operations, which is an exponential gain due to the efficiency of the quantum Fourier transform.\par
The quantum Fourier transform is an implementation of the discrete Fourier transform over amplitudes of quantum states. It offers no speed ups when used in computing Fourier transforms of classical data, since the amplitudes cannot be accessed directly by measurement. Moreover, there is no known generalized efficient way of preparing the initial state to be Fourier Transform. What this means is that the uses of the QFT are not in the straightforward way of calculating discrete Fourier transforms, but in the form of algorithms, such as \textit{phase estimation}, that take advantage of it's properties. This transform can be described as the following operation over an orthonormal basis $\ket{0}, \ket{1},\cdots,\ket{N-1}$
\begin{equation}
	QFT(\ket{j}) = \frac{1}{\sqrt{N}} \sum_{k=0}^{N-1} e^{\frac{2\pi i j k}{N}} {\ket{k}},
	\label{eq:qftGeneral}
\end{equation}
%TODO: Melhorar isto?
where $N = 2^n$. With a little bit of algebra, this can be rewritten as a product
\begin{equation}
	\begin{split}
		\frac{1}{\sqrt{N}} \sum_{k=0}^{N-1} e^{\frac{2\pi i j k}{2^n}} \ket{k} & = 
	\frac{1}{\sqrt{N}} \sum_{k_1=0}^1 \cdots \sum_{k_n=0}^{1} e^{2\pi i j (\sum_{l=1}^{n} k_l 2^{-l})} {\ket{k_1\cdots k_n}}  \\
																 %&= \frac{1}{\sqrt{N}} \sum_{k_1=0}^1 \cdots \sum_{k_n=0}^{1} \bigotimes_{l=1}^n e^{2\pi i j k_l 2^{-l} } \ket{k_l} \\
																 &= \frac{1}{\sqrt{N}} \bigotimes_{l=1}^{n}( \sum_{k_l=0}^{1}  e^{2\pi i j k_l 2^{-l} } \ket{k_l} ) \\
																 &= \frac{1}{\sqrt{N}} \bigotimes_{l=1}^{n}( \ket{0} +  e^{2\pi i j 2^{-l} } \ket{1} )
	\end{split}
\end{equation}\par
The quantum Fourier transform applied to a state as in equation \ref{eq:qftGeneral} can then be rewritten as 
\begin{equation} \label{eq:QFTCircuitRep} 
	QFT(\ket{x_1,...x_n})  =\frac{(\ket{0} + e^{2 \pi i 0.x_n} \ket{1}) (\ket{0} + e^{2 \pi i 0.x_{n-1}x_n} \ket{1})\cdots  (\ket{0} + e^{2 \pi i 0.x_1x_2\cdots x_n} \ket{1})}{2^{\frac{N}{2}}},
\end{equation}
where $x = x_1 2^{n-1} + x_2 2^{n-2} + \cdots + x_n 2^0$ and the notation $0.x_1x_{l+1}\cdots x_n$ represents the binary fraction $\frac{x_l}{2i^0}+\frac{x_{l+1}}{2^1} \cdots \frac{x_m}{2^{m-l+1}}$. This is a very useful representation because it makes constructing an efficient circuit much simpler, as can be seen in figure \ref{fig:QFTCircuit}. However, the circuit implementation of the QFT requires exponentially smaller phase-shift gates as the number of qubits increases. This can be somewhat mitigated by eliminating the smaller phase-shitft gates at the cost of some accuracy, as was shown in \cite{Coppersmith94} where he defined the \textit{approximate} quantum Fourier transform. This approximation requires only $O(n \log n)$ gates, and work by \cite{Barenco96} and \cite{Cheung2004} established lower bounds for the probability of the approximate state accurately representing the state without approximation. 
\begin{figure}[h]
        \centering
	\[ \Qcircuit @C=0.8em @R=0.7em {&\lstick{\ket{x_1}} &\qw    &\gate{H} &\gate{R_2}&\qw &\cdots &      & \gate{R_{m-1}} & \gate{R_m}&\qw&\qw      &\qw       &\qw  &\qw            &\qw         &\qw      &\qw &\qw    &\qw  &\rstick{\ket{y_1}} &\\
				   &\lstick{\ket{x_2}}    &\qw    &\qw      &\ctrl{-1}&\qw  &\qw    & \qw  &\qw             &\qw        &\gate{H}&\qw &\cdots &     &\gate{R_{m-1}} &\gate{R_m}  &\qw     &\qw &\qw &\qw     &\rstick{\ket{y_2}} &\\
				   &\lstick{\vdots}&  &\\
				   &\lstick{\ket{x_{n-1}}}&\qw    &\qw      &\qw&\qw       &\qw    &\qw & \ctrl{-3}           &\qw        &\qw &\qw     &\qw       &\qw  &\ctrl{-2}            &\qw         &\gate{H} &\gate{R_2}&\qw  &\qw    &\rstick{\ket{y_{n-1}}} &\\
                                   &\lstick{\ket{x_{n}}}&\qw    &\qw      &\qw &\qw       &\qw    & \qw  &\qw             &\ctrl{-4}       &\qw  &\qw    &\qw       &\qw  &\qw            &\ctrl{-3}        &\qw      &\ctrl{-1}&\gate{H}&\qw &\rstick{\ket{y_{n}}} &
		   }\]
        \caption{Temp}
        \label{fig:QFTCircuit}
\end{figure}
\par
The rotation $R_k$ in figure \ref{fig:QFTCircuit} is defined as the controlled version of 
\begin{equation}
	R_k = \begin{pmatrix}
		&1 & 0 & \\
		&0 & e^{\frac{2\pi i}{2^k}}
              \end{pmatrix}.
\end{equation}
To verify that this circuit is the QFT, consider the state $\ket{x_1\cdots x_n}$ as input. Applying the Hadamard gate on the first qubit produces the state
\begin{equation}
	H \ket{x_1 \cdots x_n} = \frac{1}{\sqrt{N}} ( \ket{0} + e^{2\pi i 0.x_1} \ket{1}) \ket{x_1\dots x_n}. 
\end{equation}
The next operation is the rotation $R_2$, controlled by the second qubit, resulting in state
\begin{equation}
	\frac{1}{\sqrt{N}} ( \ket{0} + e^{2\pi i 0.x_1x_2} \ket{1}) \ket{x_1\dots x_n}.
\end{equation}
Applying the succesive rotations up to $R_n$ appends an extra bit to the phase of the first $\ket{1}$, ultimately becoming
\begin{equation}
	\frac{1}{\sqrt{N}} ( \ket{0} + e^{2\pi i 0.x_1x_2\cdots x_n} \ket{1}) \ket{x_1\dots x_n}.
\end{equation}
A similar process is applied to the second qubit, and at the end of the rotations the state is
\begin{equation}
	\frac{1}{\sqrt{N}} ( \ket{0} + e^{2\pi i 0.x_1x_2\cdots x_n} \ket{1})  ( \ket{0} + e^{2\pi i 0.x_2\cdots x_n} \ket{1})\ket{x_1\dots x_n},
\end{equation}
and the successive application of this process to the remaining qubits results in state
\begin{equation}
	\frac{1}{\sqrt{N}} ( \ket{0} + e^{2\pi i 0.x_1x_2\cdots x_n} \ket{1})  ( \ket{0} + e^{2\pi i 0.x_2\cdots x_n} \ket{1})\cdots ( \ket{0} + e^{2\pi i 0.x_n} \ket{1})\ket{x_1\dots x_n},
\end{equation}
confirming that this is indeed the Fourier transform derived in equation \ref{eq:QFTCircuitRep} minus the order of the qubits, which is reversed. It also shows that the QFT is unitary, since all operations in the circuit are unitary.\par
Counting the number of gates on the circuit, one can conclude that the first qubit will have $1$ Hadamard gate followed by $n-1$ controlled rotations. The second qubit is another Hadamard followed by $n-2$ controlled rotations. After $n$ qubits, the total number of gates will be $\frac{n(n+1)}{2}$. This means the circuit provides a $O(n^2)$ algorithm, compared to the fastest classical algorithm, the \textit{Fast Fourier Transform} which requires $O(n2^n)$ operations. This is an exponential gain, which can be improved upon at the cost of accuracy, but it's not a replacement for calculating classical Fourier transforms for the aforementioned reasons. 

\end{document}

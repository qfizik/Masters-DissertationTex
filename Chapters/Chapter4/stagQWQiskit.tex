\documentclass[../../dissertation.tex]{subfiles}
\begin{document}

As was seen in section \ref{sec:chap3StagWalk}, the elements of each
tesselation of a discretely numbered cycle can be described by states
\begin{equation}
        \ket{\alpha_x} = \frac{\ket{2x} + \ket{2x+1}}{\sqrt{2}}
\end{equation}
\begin{equation}
        \ket{\beta_x} = \frac{\ket{2x+1}+\ket{2x+2}}{\sqrt{2}}.
\end{equation}
These states allow the construction of the Hamiltonians
\begin{equation}
        H_\alpha = 2\sum_{x=-\infty}^{+\infty}\ket{\alpha_{x}}\bra{\alpha_x} - I
\end{equation}
\begin{equation}
        H_\beta = 2\sum_{x=-\infty}^{+\infty}\ket{\beta_{x}}\bra{\beta_x} - I
\end{equation}
as was seen in equations \ref{eq:stagSimulHalpha} and \ref{eq:stagSimulHbeta}.
As was shown by \cite{acasiete2020}, these operators cane be rewritten in
matrix form 
\begin{equation}
	H_\alpha = I \otimes X
\end{equation}
%TODO: ESCREVER EQUACAO MATRICIAL DO HBETA
\begin{equation} 
	H_\beta = 
	\begin{pmatrix}
	0 & \cdots & 1\\
	\vdots & H_\alpha & \vdots\\
	1 & \cdots & 0
	\end{pmatrix}
\end{equation} 
which are very useful representations when constructing the circuit.\par
As was shown in equation \ref{q:stagSimulUniOp}, the unitary evolution operator is
%TODO: DESCOBRIR SE A OS US ESTAO NA ORDEM CORRETA. 
\begin{equation}
	U = e^{i\theta H_\beta}e^{i\theta H_\alpha} = U_\beta U_\alpha, 
\end{equation} 
and knowing that
%TODO: TENTAR DESCOBRIR MELHOR MANEIRA DE REPRESENTAR ISTO. SERA QUE PRECISO DA MATRIZ TODA?  
\begin{equation} 
	R_x(\theta) = e^{\frac{-i\theta X}{2}}, 
\end{equation} 
then each of the operators associated with the different tesselation
Hamiltonians can be written as
\begin{equation} 
	U_\alpha = I \otimes R_x(\theta) 
\end{equation}
%TODO: Escrever Beta
\begin{equation} 
	U_\beta = 
	\begin{pmatrix}
	\cos{\theta} & \cdots & -i\sin{\theta}\\
	\vdots & U_\alpha & \vdots\\
	-i\sin{\theta}& \cdots & \cos{\theta}
	\end{pmatrix}
\end{equation}
Notice that $U_\beta$ is simply a permutation of $U_\alpha$ and can be rewritten as
\begin{equation}
	U_\beta = P^{-1} U_\alpha P
\end{equation}
where $P = \sum_x \ket{x+1}\bra{x}$. Remember from equation
\ref{eq:shiftMatrixQiskit} and figure \ref{fig:coinedIncrement} that these
permutation operators can be implemented as increment and decrement gates, defined in the work of \cite{douglaswang07}. Therefore, the circuit for the
staggered quantum walk on the line can be constructed as is shown in figure
\ref{fig:stagQWCirc}. 
\begin{figure}[!h]
	\[ \Qcircuit @C=1em @R=0.7em {   & & && \mbox{Repeat for steps} & &\\ \\
	               &       & \qw & {/^{\otimes n-1}}\qw      & \multigate{1}{\mbox{INCR}}&\qw &  \multigate{1}{\mbox{DECR}} & \qw \\
            	   &       &\qw & \gate{R_x(2\theta)}    & \ghost{INCR} &\gate{R_x(2\theta)}        & \ghost{DECR} & \qw \gategroup{3}{4}{4}{7}{.8em}{--}
		          } \]
	\caption{Staggered quantum walk circuit}
	\label{fig:stagQWCirc}
\end{figure}\par

The next step is to implement the circuit in Qiskit, in order to test it in a real quantum computer. Here, the walk will take place in a cyclic graph with $8$ elements, meaning 3 qubits will be required as is shown in figure \ref{fig:stagQWCircuitQistkit}.
\begin{figure}[!h]
	\centering
	\includegraphics[scale=0.32]{img/Qiskit/StaggeredQW/Circuits/circStagQW_N3_S3.png}
	\caption{Temp} 
	\label{fig:stagQWCircuitQistkit}
\end{figure}\par
\begin{figure}[!h]
	\centering
	\includegraphics[scale=0.40]{img/Qiskit/StaggeredQW/StagQW_N3_S0123.png}
	\caption{Temp} 
	\label{fig:stagQWQiskitDist}
\end{figure}
The circuit starts with a Pauli-X gate in the third qubit so that $\ket{\psi_0}
= \ket{4}$. The following operation is a rotation in the X basis, where $\theta
= \frac{\pi}{3}$ since it was seen in figure \ref{fig:stagQWSimulMultTheta}
that this value of $\theta$ maximizes the propagation of the walk. Finally,
$U_\beta$ is applied, making use of the increment and decrement gates defined
in figures \ref{fig:incrCircuitQistkit} and \ref{fig:decrCircuitQistkit}. Note that now, because this model is does not require a coin, these gates do not need to be controlled, meaning that one less qubit is needed for the walk and one less multi-controlled NOT gate will be implemented, making it more NISQ computer friendly. This procedure is repeated $3$ times, and the resulting probability distributions
after measurement can be seen in figure \ref{fig:stagQWQiskitDist}. 
\par
Analyzing the figure, it is clear that this model is much better suited for running in a NISQ computer. Even though the probability distribution was somewhat affected by noise, the dynamics of the walk can be seen in the Toronto backend experiment. Highest fidelity was achieved for $2$ steps, with a value of approximately $0.95$, and the remaining steps ranging from $0.91$ to $0.92$. Again, it may seem counter-intuitive that a higher fidelity is achieved for a larger circuit, compared to $0$ steps for example, but this is again due to the balance of circuit size and spread of the probability distribution. Nevertheless, because this discrete model requires ever more operations with the increase of steps, it will eventually become intractable for NISQ technology. For this reason, the next model studied in this work is one where the circuit will be constant in time.

\end{document}

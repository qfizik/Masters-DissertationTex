\documentclass[../../dissertation.tex]{subfiles}
\begin{document}

\subsection{Grover's Algorithm}
As discussed in section \ref{sec:GrovSearchSimul}, Grover's algorithm is a
quantum process to address unstructured search problems. Consider the case of
finding element $x_0$ out of an unordered list of size $N$. For the worst case
scenario, a classical algorithm would need to check every element of the list,
therefore requiring $N$ steps.\par

The first stage of Grover's algorithm is to create an uniform superposition of
all states in the system
\begin{equation}
	\ket{\psi_0}  = \frac{1}{\sqrt{N}}\sum_{x=0}^{N-1} \ket{x}.
\end{equation}\par

The next stage is the application of the Grover iteration process, which starts
with an oracle that adds a negative phase to the solution states
\begin{equation}
        \mathcal{O}\ket{x} = (-1)^{f(x)}\ket{x}.
\end{equation}
This operator can be seen as an identity matrix with negative entries
corresponding to the solution states. The operator can be rewritten as 
\begin{equation}
	\mathcal{O} = I - 2\sum_{m\in M} \ket{m}\bra{m}.
	\label{eq:groverQiskitOracle}
\end{equation}
where $I$ is the identity matrix and $M$ is a set of solutions where $f(m) =
1$, and $0$ otherwise. The matrix associated with this operator is
\begin{equation}
	\mathcal{O} = 
	\begin{pmatrix}
		(-1)^{f(0)} & 0 & \cdots & 0\\
	        0 & (-1)^{f(1)} & \cdots & 0\\ 
	        \vdots & 0 &  \ddots & \vdots\\ 
		0 & 0 & \cdots &  (-1)^{f(N-1)}
	\end{pmatrix}.
	\label{eq:oracleMatrixQiskit}
\end{equation}\par

%TODO: Mencionar que o grover se extende a toda a categoria de problemas que envolvam aquela matriz.
The second part of the iteration is an amplitude amplification process through the diffusion operator 
\begin{equation}
        %TODO: Decidir se mantenho os H's.
        \mathcal{D} = (2\ket{\psi_0}\bra{\psi_0} - I) = H^{\otimes n}(2\ket{0}\bra{0} - I)H^{\otimes n}.
	\label{eq:groverQiskitDiffusion}
\end{equation}
The unitary operator that describes the Grover iteration process will then be
\begin{equation}
        \mathcal{U} = \mathcal{D}\mathcal{O}.
\end{equation}\par

As previously discussed, this iteration process will be repeated several times,
depending on the number of elements. Optimal probability of success in finding a
single solution will be reached after $\floor{\frac{\pi}{4}\sqrt{N}}$ steps,
and $\floor{\frac{\pi}{4}\sqrt{\frac{N}{K}}}$ for $K$ solutions, which amounts to a
quadratic gain when compared to the classical case. The general Grover circuit
can then be constructed as shown in figure \ref{fig:groverSearchCircuit}.
\begin{figure}[!h]
	\[ \Qcircuit @C=1.8em @R=1.5em {& & & && \mbox{ Repeat $O(\sqrt{N})$ times}  & & &\\
	& & {/^{\otimes n}} \qw &\gate{H}  & \gate{\mathcal{O}} &  \gate{D} & \qw &\qw \gategroup{2}{5}{2}{6}{.8em}{--}
		          } \]
	\centering
	\caption{General circuit for the Grover search.}
	\label{fig:groverSearchCircuit}
\end{figure}\par

Consider the $3$ qubit case, where $N=8$ and solution state $\ket{4}$. The
optimal number of iterations is approximately $2$. Figure
\ref{fig:groverCircuitQistkit} depicts the circuit for $3$ iterations implemented in
Qiskit.
\begin{figure}[!h]
	\centering
	\includegraphics[scale=0.30]{img/Qiskit/GroverQiskit/Circuits/GroverQiskitCirc_N3_M4_S3.png}
	\caption{Qiskit circuit for the Grover algorithm, for a search space of size $N=8$ and $3$ steps.}
	\label{fig:groverCircuitQistkit}
\end{figure}\par
The system starts with the creation of an uniform superposition state, by applying Hadamard gates to each qubit.  
\begin{figure}[!h]
	\centering
	\includegraphics[scale=0.25]{img/Qiskit/GroverQiskit/Circuits/GroverQiskitCircOracle_N3_M4_S3.png}
	\caption{Qiskit circuit of the  diagonal oracle operator in search space of size $N=8$, for marked element $\ket{m}=\ket{4}$.}
	\label{fig:groverOracleCircuitQistkit}
\end{figure}
Immediately following the barrier, the first operator of the iteration process
is the oracle, which is shown in figure \ref{fig:groverOracleCircuitQistkit}.
Because the oracle operator is simply the identity matrix with negative entries
corresponding to the solution states, it can be simply translated into a
circuit through the \textit{diagonal} function in Qiskit.\par

The last part of the iteration is the diffusion operator, whose circuit is
shown in figure \ref{fig:groverDiffCircuitQistkit}.
\begin{figure}[!h]
	\centering
	\includegraphics[scale=0.25]{img/Qiskit/GroverQiskit/Circuits/GroverQiskitCircDiff_N3_M4_S3.png}
	\caption{Qiskit circuit of the  diagonal Grover diffusion operator in search space of size $N=8$.}
	\label{fig:groverDiffCircuitQistkit}
\end{figure}
%TODO Isto esta muito fraco. 
Comparing equations \eqref{eq:groverQiskitOracle} and
\eqref{eq:groverQiskitDiffusion}, it is easy to see why figures
\ref{fig:groverOracleCircuitQistkit} and \ref{fig:groverDiffCircuitQistkit} are
so similar. The diffusion circuit will simply be the oracle circuit for state
$\ket{0}$ placed between Hadamard operations.
\begin{figure}[!h]
	\centering
	\includegraphics[scale=0.40]{img/Qiskit/GroverQiskit/GroverQiskitSearch_N3_M4_S0123}
	\caption{Probability distributions of the Grover search algorithm for several steps, in a search space of size $N=8$. The blue bar plot represents a circuit run in the QASM simulator, and the orange bar plot on IBM's Toronto backend.}
	\label{fig:groverQiskitDist}
\end{figure}\par

The results of measurement are shown in figure \ref{fig:groverQiskitDist}. As
expected, the maximum probability for the marked element was reached after $2$
iterations, on the simulator, and it decreases in subsequent steps. However,
the experimental result from the Toronto backend presents maximum probability
for the marked element for $1$ step, with a fidelity of $0.96$. The optimal
number of steps, in contrast, has a lower fidelity of $0.89$. This is because
as the number of steps increases, so does the circuit depth. Therefore, the
circuit does not achieve the maximum probability after $2$ steps due to the
effects introduced by noise.\par

Despite of them, the results are satisfactory when
taking into account the properties of NISQ computers. The following sections will
present the search problem adapted to several quantum walk models.

\subsection{Searching with a Coined Quantum Walk}
Following the structure of section \ref{sec:CoinedSearchSimul}, this section
expands the coined quantum walk model into a circuit for searching.\par 

The modified unitary evolution operator is
\begin{equation}
        U' = S (\mathcal{O} \otimes G),\label{eq:modifiedEvoCoinedQiskit}
\end{equation}
%TODO: Relembrar equacoes para cada um dos operadores? Parece me desnecessario.
as was defined in equation \eqref{eq:modifiedEvoCoined}, where $S$ is the
flip-flop shift operator, $\mathcal{O}$ is the oracle operator and $G$ is the
Grover diffusion as a coin operator.\par

Consider the case of a complete graph, where every vertex is adjacent to one
another. The general quantum circuit to implement this, as shown in figure
\ref{fig:coinedSearchCircuit}, will require $n$ qubits to represent the state
of the walker and $n$ qubits for the state of the coin.  The shift operator was
constructed based on the work of \cite{douglaswang07}, where the state of the
walker is flip-flopped with the state of the coin with a swap
operation.
\begin{figure}[!h]
	\[ \Qcircuit @C=1.8em @R=1.5em { & & & & \mbox{ Repeat $O(\sqrt{N})$ times}  &\\
	                                & {/^{\otimes n}} \qw  &\gate{H}  & \gate{\mathcal{O}} & \multigate{1}{SHIFT} & \qw &  \\
				                    & {/^{\otimes n}} \qw  & \qw & \gate{G}&   \ghost{SHIFT} & \qw \gategroup{2}{4}{3}{5}{.8em}{--}
		          } \]
	\centering
	\caption{General circuit for the search problem using the coined quantum walk model.}
	\label{fig:coinedSearchCircuit}
\end{figure}\par

This was implemented in Qiskit, for a graph of size $N=2^3=8$, which means $6$
qubits will be required. For the case of one marked element, the number of
iterations that maximizes the amplitude of the solution state is
$\floor{\frac{\pi}{2} \sqrt{N}}$. Figure
\ref{fig:coinedQWSearchCircuitQistkit} shows the circuit for 5 iterations of
the walk.
\begin{figure}[!h]
	\centering
	\includegraphics[scale=0.21]{img/Qiskit/CoinedQuantumWalk/Search/Circuits/CoinedSearchQiskitCirc_N3_M0_S5.png}
	\caption{Qiskit circuit for the search problem using the coined quantum walk model, for a complete graph of size $N=8$ and with $5$ steps.} 
	\label{fig:coinedQWSearchCircuitQistkit}
\end{figure}\par

The circuit starts in a uniform superposition of the states corresponding to
the vertices of the graph. The first step of the iteration is the oracle.
This operator flips the amplitude of the vertex state $\ket{4}$, and can be
translated into a circuit making use of the Qiskit's \textit{diagonal} function, as is shown
in figure \ref{fig:coinedQWSearchOracleCircuitQistkit}. 
\begin{figure}[!h]
	\centering
	\includegraphics[scale=0.30]{img/Qiskit/CoinedQuantumWalk/Search/Circuits/CoinedSearchQiskitCircOracle_N3_M4_S5.png}
	\caption{Qiskit circuit of the  diagonal oracle operator in the coined quantum walk search problem, for a complete graph of size $N=8$, with marked element $\ket{m}=\ket{4}$.} 
	\label{fig:coinedQWSearchOracleCircuitQistkit}
\end{figure}
It is the same oracle used in the Grover search of figure
\ref{fig:groverOracleCircuitQistkit}, but in the coined quantum walk model it
is only applied to the states associated with the position of the walker. The
states associated with the coin space of the walk will be transformed according
to Grover's diffusion of figure \ref{fig:groverDiffCircuitQistkit}, as seen
in figure \ref{fig:coinedQWSearchDiffCircuitQistkit}. 
\begin{figure}[!h]
	\centering
	\includegraphics[scale=0.30]{img/Qiskit/CoinedQuantumWalk/Search/Circuits/CoinedSearchQiskitCircDiff_N3_M4_S5.png}
	\caption{Qiskit circuit of the  diagonal diffusion operator in the coined quantum walk search problem, for a complete graph of size $N=8$.} 
	\label{fig:coinedQWSearchDiffCircuitQistkit}
\end{figure}\par

The final part of the iteration is the shift operator, as represented in figure
\ref{fig:coinedQWSearchShiftCircuitQistkit}. The flip-flop shift operator was
defined in equation \eqref{eq:chap3FlipFlop} as
\begin{equation}
        S\ket{v1}\ket{v2} = \ket{v2}\ket{v1},
        \label{eq:chap4FlipFlop}
\end{equation}
where $\ket{v1}$ represents the position of the walker and $\ket{v2}$ is the
state of the coin. Making use of the swap gate, this operator can be
implemented as in figure \ref{fig:coinedQWSearchShiftCircuitQistkit}.
\begin{figure}[!h]
	\centering
	\includegraphics[scale=0.27]{img/Qiskit/CoinedQuantumWalk/Search/Circuits/CoinedSearchQiskitCircShift_N3_M4_S5.png}
	\caption{Qiskit circuit of the  flip-flop shift operator in the coined quantum walk search problem, for a complete graph of $N=8$.} 
	\label{fig:coinedQWSearchShiftCircuitQistkit}
\end{figure}\par

Lastly, measurements were performed. The results are plotted in figure
\ref{fig:coinedSearchQiskitDist}. Maximum probability of the marked element was
reached after $4$ steps in the simulator. Extra steps reduce that probability.
The resulting probability distribution from the Toronto backend is again
unsatisfactory for the coined quantum walk model, with fidelities ranging from
$0.58$ for $4$ steps, to $0.75$ for $2$ steps. This is expected, since the
complete graph representation requires $N$ extra qubits for the coin, and swap
operations which are decomposed into $3$ CNOTs each. The optimal number of
steps that maximizes the probability of the marked element is also a
contributing factor to the size of the circuit, requiring more iterations to
achieve the same probability when compared to Grover's search discussed above. 
\begin{figure}[!h]
	\centering
	\includegraphics[scale=0.40]{img/Qiskit/CoinedQuantumWalk/Search/CoinedQiskitSearch_N3_M4_S0245}
	\caption{Probability distributions of the coined quantum walk search problem for several steps, in a complete graph of size $N=8$. The blue bar plot represents a circuit run in the QASM simulator, and the orange bar plot on IBM's Toronto backend.} 
	\label{fig:coinedSearchQiskitDist}
\end{figure}\par

As mentioned previously, other models of the quantum walks that do not require
coins or iterations will be studied in the following sections, in the context
of the searching problem. The staggered quantum walk, for example, should be
able to present better results when ran in a NISQ computer, considering the
smaller Hilbert space due to its coinless nature.

\subsection{Searching with a Staggered Quantum Walk}
As discussed in section \ref{sec:StagSearchSimul}, the staggered quantum walk
on a complete graph requires a single tessellation with associated polygon
\begin{equation}
	\ket{\alpha} = \frac{1}{\sqrt{N}} \sum_{x=0}^{N-1} \ket{x}.
\end{equation}
The Hamiltonian will then be 
\begin{equation}
	H_\alpha = 2\sum_0^1\ket{\alpha}\bra{\alpha} - I = H^{\otimes n} (2\ket{0}\bra{0} - I) H^{\otimes n} = H^{\otimes n} \mathcal{O}_0 H^{\otimes n},
\end{equation}
which is equivalent to the Grover diffusion operator. Therefore, it can be
implemented in a similar fashion. \par

The evolution operator for the staggered quantum walk on the complete graph can
then be defined as 
\begin{equation}
	U = e^{i\theta H_\alpha} = e^{i\theta(H^{\otimes n} \mathcal{O}_0 H^{\otimes n})} = H^{\otimes n} e^{i\theta\mathcal{O}_0} H^{\otimes n}.
	\label{eq:unmodEvolOperatorStagSearch}
\end{equation}
This is a very useful representation since the exponent part of the operator is
a diagonal matrix, which means that implementing the circuit in Qiskit is
straightforward.\par 

Now that the staggered quantum walk associated with the complete graph is
defined, what remains to be done is to add an oracle to the evolution operator,
as was done in equation \eqref{eq:stagSearchSimulModEvoOp},
\begin{equation}
        U' = U\mathcal{O},
        \label{eq:stagSearchQiskitModEvoOp}
\end{equation}
where
\begin{equation}
	\mathcal{O} = I_N - 2\sum_{m \in M}\ket{m}\bra{m},
\end{equation}
and $M$ is the set of marked elements.\par
The general circuit for implementing the staggered quantum walk search problem
in a complete graph is shown in figure
\ref{fig:stagSearchCircuit}. 
\begin{figure}[!h]
	\[ \Qcircuit @C=1.8em @R=1.5em { &&&&& \mbox{Repeat $O(\sqrt{N})$ times.} & &\\
	& & {/^{\otimes n}} \qw& \gate{H} &\gate{\mathcal{O}} &\gate{H}  & \gate{e^{i\theta\mathcal{O}_0}} &  \gate{H} &\qw \gategroup{2}{5}{2}{8}{.8em}{--} \\
		          } \]
	\centering
	\caption{General circuit for the search problem using the staggered quantum walk model.}
	\label{fig:stagSearchCircuit}
\end{figure}
Since only one tessellation is required, there is no need for the Suzuki-Trotter
approximation. However, several iterations will be needed in order to achieve the
maximum probability for the marked vertex. As the staggered quantum walk
search algorithm a complete graph is equivalent to Grover's algorithm, the optimum
number of steps will also be $\floor{\frac{\pi}{4}\sqrt{\frac{N}{K}}}$, where K is the
number of solutions.\par
%TODO: Definir para todas as walks N grande = numero total de elementos e n pequeno = numero de qubits.
Consider the case of $N=8$ and one marked vertex, $\ket{m}=\ket{4}$. The number of steps
that maximizes the probability of the marked element is
$\floor{\frac{\pi}{4}\sqrt{\frac{8}{1}}} = 2$. Translating to Qiskit, $n=3$
qubits will be needed and the circuit will be as in figure \ref{fig:stagSearchCircQistkit}. 
\begin{figure}[!h]
	\centering
	\includegraphics[scale=0.35]{img/Qiskit/StaggeredQW/Search/Circuits/StagSearchCircuit_N3_M0_S3.png}
	\caption{Qiskit circuit for the search problem using the staggered quantum walk model, for a complete graph of size $N=8$, with $3$ steps and a value of $\theta = \frac{\pi}{2}$.}
	\label{fig:stagSearchCircQistkit}
\end{figure}
Similar to previous examples, the circuit begins with a uniform superposition
built by the Hadamard gates. The next operation is
the oracle, which was implemented through the use of Qiskit's \textit{diagonal}
function, producing a circuit similar to the one in figure
\ref{fig:groverOracleCircuitQistkit}.\par

Next, an analogue to Grover's diffusion operator is applied, where the
operation named \textit{UniOp} is a diagonal matrix, easily translated to
Qiskit, as shown in figure \ref{fig:stagSearchUniOpCircQistkit}.  
\begin{figure}[!h]
	\centering
	\includegraphics[scale=0.40]{img/Qiskit/StaggeredQW/Search/Circuits/StagUniOpCircuit_N3_M0_S3.png}
	\caption{Qiskit circuit of the  diagonal diffusion operator in the staggered quantum walk search problem, for a complete graph of size $N=8$ and a value of $\theta = \frac{\pi}{2}$.}
	\label{fig:stagSearchUniOpCircQistkit}
\end{figure}
This circuit is very similar to the one in figure
\ref{fig:groverDiffCircuitQistkit}, the difference being that in the staggered
quantum walk search model one can control the value of $\theta$, as can be seen
in equation \eqref{eq:unmodEvolOperatorStagSearch}. This influences how fast
maximum probability of the marked element is achieved. Since Grover's algorithm
is optimal, a value of $\theta=\frac{\pi}{2}$ yields a diffusion circuit equal
to the one in figure \ref{fig:groverDiffCircuitQistkit}, implying that the
staggered quantum walk is a more general model of quantum searching.
\begin{figure}[!h]
	\centering
	\includegraphics[scale=0.40]{img/Qiskit/StaggeredQW/Search/stagSearchToronto_N3_S0123.png}
	\caption{Probability distributions of the staggered quantum walk search problem for several steps, in a complete graph of size $N=8$. The blue bar plot represents a circuit run in the QASM simulator, and the orange bar plot on IBM's Toronto backend.}
	\label{fig:stagSearchResultsToronto}
\end{figure}\par

Finally, measurement is performed and the results for several steps of the walk
are shown in figure \ref{fig:stagSearchResultsToronto}.  The circuit for each
step of the walk was run both in the \textit{QASM} simulator and IBM's backend
named Toronto. The experiment was performed with $3000$ shots in both cases,
and the probability distributions show that this model is indeed more suited
for a NISQ computer than the previous case.  However, unlike the simulator,
maximum probability of the marked vertex was achieved in $1$ step of the walk
instead of the expected $2$ steps, as in the Grover algorithm. Looking at the
$1$ step case in the simulator, one can see that the probability of vertex
$\ket{4}$ is very close to the maximum, while the circuit has about half of the
operations of the $2$ step case. This means it is not surprising that the
smaller circuit produces higher results for the probability of the marked
vertex. 
%TODO: Talvez melhorar as fidelidades com o intervalo de confianca. 
This can be further confirmed by the fidelities of each of the states, which
are approximately $0.954$ and $0.780$ for $1$ and $2$ steps, respectively.
Thus, the former circuit produces better results because of the number of
operations, even though that the latter should theoretically yield the highest
probability.\par 

Even though these results are a great improvement with respect to the algorithm using
the coined quantum walk, the staggered model is still discrete, meaning its
circuit will increase in depth as the number of steps increases. This can be
avoided by turning again to the continuous-time quantum walk, whose circuit
remains constant with time.  However, to address the searching problem, the
continuous-time model might not produce the best results, because it will need extra
iterations due to the Suzuki-Trotter approximation and it will also require
extra operations to implement the oracle, which was not the case in the pure
dynamics example of section \ref{sec:ContQiskit}.

\subsection{Searching with a Continuous-Time Quantum Walk}
As seen in section \ref{sec:ContSearchSimul}, the unitary operator
associated with the continuous time quantum walk model can be modified to
mark an element for amplitude amplification
\begin{equation}
	U'(t) = e^{iH't} = \phi(t)e^{-i(\gamma A+O)t},
	\label{eq:qiskitU'}
\end{equation}
where $\phi(t)$ is a global phase, $A$ is the adjacency matrix and the oracle defined as 
\begin{equation}
	O = \sum_{m \in M} \ket{m}\bra{m},
\end{equation}
for $M$ being the set of marked elements.\par

This section will focus on constructing and analyzing the circuit corresponding
to the continuous-time quantum walk search problem, to be performed over a
complete graph whose adjacency matrix is the following
\begin{equation}
  A = 	\begin{pmatrix}
	  0 & 1 &  \cdots & 1 & 1\\
	  1 & 0 & \cdots & 1 & 1\\
	   \vdots & \vdots  & \vdots & \vdots& \vdots\\
	  1 & 1 & \cdots & 0 & 1\\ 
	  1 & 1 & \cdots & 1 & 0\\ 
	\end{pmatrix}.
\end{equation}
which is simply a matrix with all entries set to $1$, except the diagonal.\par

The first step is to borrow the diagonal definition of the adjacency matrix
from equation \eqref{eq:qiskitContQWAdj} 
\begin{equation}
    A = F^{\dagger} \Lambda F,
    \label{eq:qiskitContSearchAdj}
\end{equation}
and use the Suzuki-Trotter expansion
\begin{equation}
	e^{i(H_0+H_1)t}=\lim_{r \rightarrow \infty}(e^{i\frac{H_0t}{r}}e^{i\frac{H_1t}{r}})^r ,
\end{equation}
to decompose the operator in equation \eqref{eq:qiskitU'} 
\begin{equation}
	e^{i(\gamma A+O)t} =\lim_{r \rightarrow \infty}(F^{\dagger} e^{i\gamma\frac{\Lambda t}{r}} F e^{i\frac{Ot}{r}})^r.
	\label{eq:suzTrotter}
\end{equation}
This can be easily translated into a circuit scheme as in figure
\ref{fig:contSearchCircuit}. 
\begin{figure}[!h]
	\[ \Qcircuit @C=1.8em @R=1.5em {& & & &  \mbox{Repeat r times} & &\\
	&  {/^{\otimes n}} \qw &\gate{\mbox{H}}  & \gate{\mbox{$e^{i\frac{Ot}{r}}$}} & \gate{\mbox{F}} &\gate{\mbox{$e^{i\gamma\frac{\Lambda t}{r}}$}} & \gate{\mbox{F$^{\dagger}$}} &\qw \gategroup{2}{4}{2}{7}{.8em}{--}
		          } \]
	\centering
 	\caption{General circuit for the search problem using the continuous-time quantum walk model.}
	\label{fig:contSearchCircuit}
\end{figure}\par

Consider the case of a graph of size $N=2^3=8$ and trotter number of $r=1$. The
corresponding Qiskit circuit is shown in figure \ref{fig:contSearchCircuit}.
\begin{figure}[!h]
	\centering
	\includegraphics[scale=0.30]{img/Qiskit/ContQuantumWalk/Search/Circuits/circContSearch_N3_S2.png}
	\caption{Qiskit circuit for the search problem using the continuous-time quantum walk model, for a complete graph of size $N=8$, time $t$ and a value of $\gamma = \frac{1}{8}$.}
	\label{fig:contSearchCircQistkit}
\end{figure}
The system starts out in an uniform superposition followed by the application
of the oracle operator as can be seen in figure
\ref{fig:contSearchOracleCircQistkit}. Note that the circuit was obtained with
Qiskit's \textit{diagonal} function that takes the diagonal entries of the
operator corresponding to the oracle defined in equation
\eqref{eq:suzTrotter}.\par

The next operation to be applied is the QFT but, because a complete graph is
considered, the AQFT can be used with a degree of $m=2$, which means the
circuit will simply amount to Hadamard transforms, thus reducing its depth.
\begin{figure}[!h]
	\centering
	\includegraphics[scale=0.30]{img/Qiskit/ContQuantumWalk/Search/Circuits/circOracle_N3_S2.png}
	\caption{Qiskit circuit of the  diagonal oracle operator in the continuous-time quantum walk search problem, for a complete graph of size $N=8$, marked element $\ket{m}=\ket{4}$ and time $t=\frac{\pi}{2} \sqrt{8}$.}
	\label{fig:contSearchOracleCircQistkit}
\end{figure}\par

Then, the operator associated with the adjacency matrix is shown in figure
\ref{fig:contSearchAdjCircQistkit}. Since $A$ is the diagonal adjacency matrix
of a complete graph, it is easily implemented using the aforementioned
\textit{diagonal} function.\par
\begin{figure}[!h]
	\centering
	\includegraphics[scale=0.30]{img/Qiskit/ContQuantumWalk/Search/Circuits/circAjd_N3_S2.png}
	\caption{Qiskit circuit of the  diagonal operator associated with the adjacency matrix in the continuous-time quantum walk search problem, for a complete graph of size $N=8$, marked element $\ket{m}=\ket{4}$, time $t=\frac{\pi}{2} \sqrt{8}$ and $\gamma = \frac{1}{8}$.}
	\label{fig:contSearchAdjCircQistkit}
\end{figure}
Finally, the results of circuit measurement can be seen in figure
\ref{fig:contSearchResultCircQistkit}. Even though the circuit depth does not
scale up with time, introducing the oracle operation to the continuous-time
quantum walk model appears to make the circuit hard to run in a NISQ computer.
Note that, theoretically, $2$ iterations of the Suzuki-Trotter expansion are
needed for maximum probability of the marked vertex to be achieved in optimal
time.  In practice, however, greater probability for this element is achieved
when only $1$ iteration is performed since the smaller circuit seems to produce
a greater fidelity of $0.71$, when compared to a fidelity of $0.59$ for the $2$
iteration case. The approximate quantum Fourier transform has also a positive
impact, increasing fidelity from $0.60$ when the full QFT circuit is used, to
the aforementioned $0.71$ when only Hadamard transformations are performed.
\begin{figure}[!h]
	\centering
	\includegraphics[scale=0.40]{img/Qiskit/ContQuantumWalk/Search/ContQW_N3_S2.png}
	\caption{Probability distributions of the continuous-time quantum walk search problem for several time intervals, in a complete graph of size $N=8$. The blue bar plot represents a circuit run in the QASM simulator, and the orange bar plot on IBM's Toronto backend.}
	\label{fig:contSearchResultCircQistkit}
\end{figure}\par

In conclusion, when comparing the different quantum walk models studied in
previous sections, the coined quantum walk seems to be the least suited for
supporting a search algorithm in NISQ computers. Then comes the continuous-time
model which has the advantage of not scaling with time, but produces a circuit
slightly too large to achieve a fidelity which rivals the staggered quantum
walk. The latter appears to have the best performance because, even though its discrete
nature requires several iterations, the circuit depth does not appear to
introduce a drastic amount of noise for the $N=8$ case. 

\end{document}

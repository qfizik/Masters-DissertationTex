\documentclass[../../dissertation.tex]{subfiles}
\begin{document}

As was seen in section \ref{sec:chap3Contwalk}, the unitary evolution operator of this model is defined as
\begin{equation}
        U(t) = e^{-iHt} = e^{i(-\gamma L)t} = e^{-i\gamma(A+D)t}.
\end{equation}
Considering a regular graph, this operator can be rewritten as 
\begin{equation}
	U(t) = \phi(t) e^{-i\gamma(A)t},
\end{equation}
where $\phi(t)$ is a global phase and $A$ is the adjacency matrix associated with the graph.\par
Here, the study will focus on the circuit implementation of this walk in a cyclic graph, and a different approach was used to define the adjacency matrix. This approach relies on the concept of a circulant graph, which are a class of graphs defined by a circulant matrix such that
\begin{equation}
A = 
	\begin{pmatrix}
		c_0&c_{N-1}& \cdots&c_3&c_2 \\
		c_1&c_0& c_{N-1}& &c_{3} \\
		\vdots & c_1 & c_0 &\ddots & \vdots\\
		c_{N-2}& & \ddots&\ddots &c_{N-1}\\
		c_{N-1} & c_{N-2} & \cdots & c_1 & c_0\\
	\end{pmatrix}.
\label{eq:adjCirculant}
\end{equation}
In order to generate the proper circulant graphs, restrictions on this matrix are needed. Firstly, $c_0=0$, since self-loops are not part of the structure. Secondly, the matrix must be symmetric, therefore $c_{n-j} = c_j$.\par
These matrices can be fully described by the first column of the matrix 
\begin{equation}
	v_1 = [c_{0},c_{1}, \cdots, c_{N-2}, c_{N-1}]^T 
\end{equation}
with a discrete convolution operator performing cyclic permutations of $c$, on each column. For example,
\begin{equation}
	D v_1 = [c_{N-1}, c_{0}, \cdots, c_{N-3}, c_{N-2}]^T = v_2.
\end{equation}
More specifically, for the cycle case
\begin{equation}
	D v_1 = D [0, 1 ,0 , \cdots, 0, 1]^T =[1, 0, 1, 0, \cdots, 0, 0]^T = v_2.
\end{equation}
%TODO: Estas definicoes estao mal. 
The eigenvalues of a circulant matrix can be found by
\begin{equation}
\lambda_p = c_0 + \sum_{q=1} c_{N-q} \omega^{pq},
\end{equation}
and the eigenvectors are 
\begin{equation}
\ket{\varphi_p} = \frac{1}{\sqrt{n}} \sum_{q=0}^{n-1} \omega^{pq}.
\end{equation}
This given, it is possible to construct an operator that diagonalizes the circulant matrix through the eigenvectors, which is useful for constructing the circuit. For this purpose, the Quantum Fourier Transform can be used and it is defined as 
\begin{equation}
F = \frac{1}{\sqrt{N}} \sum_{p,q} \omega^{pq} \ket{p}\bra{q}.
\end{equation}
The adjacency matrix of a circulant graph can then be diagonalized such that
\begin{equation}
    A = F^{\dagger} \Lambda F,
    \label{eq:qiskitContQWAdj}
\end{equation}
where $\Lambda$ is a diagonal operator that encodes the eigenvalues
\begin{equation}
\Lambda = \sum_{j} \lambda_j \ket{j}\bra{j}.
\end{equation}\par
%TODO: Mostrar o raciocinio da algebra.
The unitary operator of the walk can then be rewritten as
\begin{equation}\label{eq:diagUniOpCont}
    U = F^{\dagger}e^{i\gamma \Lambda t} F
\end{equation}
where
\begin{equation}
    e^{i\gamma \Lambda t} = \sum_{j} e^{i\gamma \lambda_j t} \ket{j}\bra{j}.
\end{equation}
This representation is very easily translatable to a quantum circuit, as can be seen in figure \ref{fig:contCircuit}.
\begin{figure}[!h]
	\[ \Qcircuit @C=1.8em @R=1.5em {
	& & {/^{\otimes n}} \qw &\gate{QFT_m}  & \gate{e^{i\gamma\Lambda t}} &  \gate{QFT^{\dagger}_m} &\qw &\qw 
		          } \]
	\centering
	\caption{Temp}
	\label{fig:contCircuit}
\end{figure}\par
%TODO: Explicar a diagonal()?
The circuit can now be constructed making use of the \textit{diagonal} function provided by Qiskit, which decomposes diagonal operators based on the method presented in theorem 7 of \cite{Shende06}. The other tool used was the Quantum Fourier Transform, also provided by the Qiskit package. Figure \ref{fig:contQWCircuitQistkit} shows the implementation of the circuit for 3 qubits or $2^3=8$ graph nodes and $t=3$.

\begin{figure}[!h]
	\centering
	\includegraphics[scale=0.25]{img/Qiskit/ContQuantumWalk/Circuits/circContQW_N3_S1.png}
	\caption{Temp} 
	\label{fig:contQWCircuitQistkit}
\end{figure}
%TODO: Perceber como explicar a QFT.
The circuit for the Quantum Fourier transform is well known and presented in figure \ref{fig:qftCircuitQiskit}. The circuit associated with $F^{\dagger}$ is similarly constructed by negating the previous figure's rotations.\par
\begin{figure}[!h]
	\centering
	\includegraphics[scale=0.28]{img/Qiskit/ContQuantumWalk/Circuits/circQft_N3_S1.png}
	\caption{Temp} 
	\label{fig:qftCircuitQiskit}
\end{figure}
The circuit associated with the diagonal operator is show in figure \ref{fig:diagCircuitQiskit}. Furthermore equation \ref{eq:diagUniOpCont} says that time is simply a constant inside the exponential, which means that the diagonal operator's circuit will not need extra gates when increasing time, it will only need different rotations and differ in global phase. 
\begin{figure}[!h]
	\centering
	\includegraphics[scale=0.50]{img/Qiskit/ContQuantumWalk/Circuits/circDiag_N3_S1.png}
	\caption{temp}
	\label{fig:diagCircuitQiskit}
\end{figure}
This is an advantage when comparing to the previous model, seen in figure \ref{fig:coinedQWCircuitQistkit}, where each extra step required another increment and decrement gates.  
Finally, the circuit was measured, and the result can be seen in figure \ref{fig:contQWQiskitDist}.
\begin{figure}[!h]
	\centering
	\includegraphics[scale=0.4]{img/Qiskit/ContQuantumWalk/ContQW_N3_S0123.png}
	\caption{Temp} 
	\label{fig:contQWQiskitDist}
\end{figure}
The dynamics of the walk, when ran on the Toronto backend, is closer to the simulation than the previous examples. As expected, since the size of the circuit does not scale with time, a fidelity of approximately $0.97$ was achieved for $t=3$. For $t=0,1,2$, the respective fidelities were $0.96, 0.90, 0.92$, which are small improvements when compared to the staggered quantum walk, but that stay relatively constant with increasing time, making this type of circuit well suited for studying the dynamics of the continuous-time quantum walk.

\subsubsection{Further Experiments}
Another improvement can be made to this model by use of the approximate quantum Fourier transform (AQFT). The AQFT, proposed by \cite{Coppersmith94}, is achieved through the modification of the regular quantum Fourier transform circuit, as defined in annex \ref{sec:chapQFT}, by removing the phase-shift operations between the most distant qubits. This can be implemented in Qiskit by providing the \textit{QFT} function the approximation degree.\par
Each experiment is performed 10 times, with $3000$ shots each, in order to extract substantial statistical data, using the confidence interval of $95\%$. The average fidelity between the ideal $p(x,t)$ and the experimental $q(x,t)$ distributions is again calculated using
\begin{equation}
    F(p,q) = \frac{1}{10}\sum_{i = 1}^{10} \sum_{x=0}^{N-1} \sqrt{p(x,t)q(x,t)},
    \label{eq:avgFid}
\end{equation}
which is a simple modification of equation \ref{eq:avgFid}.\par
For this section, not only is the cyclic graph considered, but a considerable number of other circulant graph as is shown in figure \ref{fig:circulantGraphs}. The numbering of the graphs follows the rule that $G_k$ will have entries $c_k, \cdots, c_1 = 1$ and $c_{n-k},\cdots,c_{n-1} = 1$, and the remaining elements are $0$. This way, it is possible to systematically construct circulant graphs varying from sparse to dense.  
\begin{figure}[!h]
  \centering
  \begin{subfigure}[t]{.23\textwidth}
    \centering
    \includegraphics[width=\linewidth]{img/Qiskit/ContQuantumWalk/Graphs/graph0.png}
    \caption{$G_1$}
  \end{subfigure}
  %\hfill
  \begin{subfigure}[t]{.23\textwidth}
    \centering
    \includegraphics[width=\linewidth]{img/Qiskit/ContQuantumWalk/Graphs/graph1.png}
    \caption{$G_2$}
  \end{subfigure}
  %  \medskip
  \begin{subfigure}[t]{.23\textwidth}
    \centering
    \includegraphics[width=\linewidth]{img/Qiskit/ContQuantumWalk/Graphs/graph2.png}
    \caption{$G_3$}
  \end{subfigure}
  %\hfill
  \begin{subfigure}[t]{.23\textwidth}
    \centering
    \includegraphics[width=\linewidth]{img/Qiskit/ContQuantumWalk/Graphs/graph3.png}
    \caption{$G_4$}
  \end{subfigure}
  \caption{Circulant graphs $G_k$ for $N=8$ elements.}
  \label{fig:circulantGraphs}
\end{figure}

%TODO: Modificar o texto para impessoal e nao ficar exatamente igual ao artigo.
%TODO: Alterar formatacao das tabelas.
Starting with the smaller $N = 2^2 = 4$ case, two non-isomorphic circulant graphs are possible to construct. $G_1$ corresponds to the cycle graph, and $G_2$ to the complete graph. Table \ref{tab:N2TRT} shows the achieved average fidelities, which were calculated with \eqref{eq:avgFid}. Comparing to the complete graph case presented in \cite{Qiang2016} , where they obtained a fidelity of $0.967 \pm 0.003$, it is possible to see that the implementation presented here slightly outperforms it in terms of fidelity.
%\vspace{-0.2cm}
\begin{table}[!h]
\centering
%\resizebox{120}{25}{
\begin{tabular}{| l | l | l |}
\hline
m$\setminus G$ & $G_1$               & $G_2$  \\ \hline
0   & 0.98 $\pm$ 0.01 & 0.99 $\pm$ 0.01  \\\hline
1   & 0.98 $\pm$ 0.02 & 0.993 $\pm$ 0.006   \\\hline
\end{tabular}
%}
\caption{Fidelity of quantum state with N=4, backend \textit{Toronto}, and t=1.}
\label{tab:N2TRT}
\end{table}\par
%\vspace{-0.3cm}
In the $N = 2^3 = 8$ case, table \ref{tab:N3TRT} shows that state fidelity is greater as graph connectivity increases, and as $m$ increases. This is due to the fact that a greater $m$ implies a smaller circuit, but also an increase in error. However, graphs have less distinct eigenvalues as they are more connected, which means that a higher degree of approximation of the QFT will generally introduce less error in this cases, while keeping circuit depth lower.
\begin{table}[!h]
\centering
%\resizebox{200}{30}{
\begin{tabular}{| l | l | l | l | l |}
\hline
m$\setminus G$ & $G_1$                & $G_2$                & $G_3$                & $G_4$    \\\hline
0   & 0.80 $\pm$ 0.01 & 0.92 $\pm$ 0.02     & 0.968 $\pm$ 0.007  & 0.965 $\pm$ 0.006 \\\hline
1   & 0.894 $\pm$ 0.007 & 0.95 $\pm$ 0.01   & 0.98 $\pm$ 0.01    & 0.973 $\pm$ 0.008 \\\hline
2   & 0.852 $\pm$ 0.009 & 0.955 $\pm$ 0.004 & 0.985 $\pm$ 0.003  & 0.990 $\pm$ 0.002 \\\hline
\end{tabular}
%}
\caption{Fidelity of quantum state with N=8, backend \textit{Toronto}, and t=1.}
\label{tab:N3TRT}
\end{table}
\vspace{-0.2cm}\par
\begin{table}[!h]
\centering
\resizebox{\columnwidth}{35}{
\begin{tabular}{|l|l|l|l|l|l|l|l|l|l|}
\hline
m$\setminus G$ & G1               & G2    & G3    & G4 & G5    & G6    & G7    & G8   \\ \hline
0   &   0.47 $\pm$ 0.03   &   0.61 $\pm$ 0.02     &   0.78 $\pm$ 0.02       & 0.86 $\pm$ 0.01 &   0.86 $\pm$ 0.01 &   0.70 $\pm$ 0.04   &   0.54 $\pm$ 0.03 &   0.49 $\pm$ 0.04  \\ \hline
1   & 0.50 $\pm$ 0.03     & 0.63 $\pm$ 0.03        & 0.79 $\pm$ 0.03        & 0.87 $\pm$ 0.02 & 0.85 $\pm$ 0.03 & 0.70 $\pm$ 0.03 & 0.55 $\pm$ 0.05 & 0.50 $\pm$ 0.04  \\ \hline
2   & 0.55 $\pm$ 0.03     & 0.71 $\pm$ 0.03     & 0.83 $\pm$ 0.02     & 0.90 $\pm$ 0.01 & 0.89 $\pm$ 0.02 & 0.75 $\pm$ 0.02 & 0.62 $\pm$ 0.04 & 0.59 $\pm$ 0.06  \\ \hline
3   & 0.60 $\pm$ 0.03     & 0.70 $\pm$ 0.02     & 0.85 $\pm$ 0.01     & 0.92 $\pm$ 0.01 & 0.91 $\pm$ 0.01 & 0.80 $\pm$ 0.03 & 0.71 $\pm$ 0.04 & 0.69 $\pm$ 0.04   \\\hline
\end{tabular}
}
\caption{Fidelity of quantum state with N=16, backend \textit{Toronto}, and t=1.}
\label{tab:N4TRT}
\end{table}
Finally,  table \ref{tab:N4TRT} presents the $N = 2^4 = 16$ case. Here, the behaviour is similar to the previous case up to graph $G_5$, meaning that higher graph connectivity and larger $m$ will result in higher fidelity. However, graphs $G_6$, $G_7$ and $G_8$, even though are highly connected and have relatively low depth, present lower fidelity. This seems to contradict the results in table \ref{tab:N3TRT}, however this behavior can be explained due to the fact that the probability distribution of the dynamics of the walk on these structures is highly concentrated in a small number for vertices. Considering that the size of the circuit starts to push the limits of NISQ computers, it is expected that the spreading out of the probability distribution due to the effects of noise results in a lower fidelity. Nonetheless, the increase of $m$ still has a positive impact on the fidelity of these circuits, which means that the reduction in circuit size will indeed produce better results.
\end{document}

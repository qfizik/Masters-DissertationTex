\documentclass[../../dissertation.tex]{subfiles}
\begin{document}

This chapter is dedicated to the construction of quantum walk circuits using
IBMs software, \textit{Qiskit}, whose code is made publicly available in the
Github repository\footnote{https://github.com/JaimePSantos/QWQiskit}. It is
composed of two parts. The first one includes sections \ref{sec:CoinedQiskit}
through \ref{sec:ContQiskit} present the circuits for the dynamics of each
previously studied quantum walk model. The second part corresponds to section
\ref{sec:SearchProblemsQiskit}, where the search algorithm is studied for each
model.\par 

The first part of this chapter begins with section \ref{sec:CoinedQiskit},
where the coined quantum walk implementation based on the work of
\cite{douglaswang07} can be found. Here, the line graph is considered again,
whose shift operation is composed of increment and decrement gates constructed
with generalized CNOT gates. As was seen in previous chapters, this model
requires an extra qubit for the space of the coin that, when combined with all
the operations required to implement the generalized CNOT gates, results in a
circuit far too deep to be implemented in current NISQ computers, as was
concluded at the end of this section. The staggered quantum walk circuit is
presented in the following section, based on the work of \cite{acasiete2020}.
This implementation still uses the notion of increment and decrement
operations. However, since a coin was not required, the resulting circuit
becomes much more NISQ-friendly. The first part is then concluded with section
\ref{sec:ContQiskit}, where the continuous-time quantum walk circuit is
implemented. The best results for the dynamics of the quantum walk were
obtained for this case, due to the circulant graph approach that, unlike the
previous discrete models, does not require extra iterations of the algorithm to
represent time. Work by \cite{qiang2016} firstly presented the circulant graph
definition of this model limited to the complete graph case, and the final part
of this section greatly expands on this work by means of statistical analysis
of the impact of the approximate quantum Fourier transform on a large
collection of circulant graphs.\par

Finally, section \ref{sec:SearchProblemsQiskit} closes this chapter with the
implementation of circuits for the search problem using quantum walks. It
starts with the introduction of quantum searching by means of the circuit
associated with Grover's algorithm, followed by the coined quantum walk, which
produces the worst results due to the requirement of $2n$ qubits and swap gates
for the representation of the complete graph. Surprisingly, the best results
were achieved for the staggered quantum walk search problem, since it produced
the circuit with the least operations. The continuous-time quantum walk search
was expected to produce the best results in this section also, but the
introduction of the oracle together with the requirement of the Suzuki-Trotter
expansion resulted in a circuit somewhat impacted by noise.
\end{document}

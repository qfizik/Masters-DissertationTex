\documentclass[../../dissertation.tex]{subfiles}
\begin{document}

Consider the example of a quantum walker on a discretely numbered cycle. It was seen that the evolution operator associated with such a system is, as was defined in equation \ref{eq:coinedUnmarkedOperator}
\begin{equation}
	   U = S(C\otimes I), 
           \label{eq:coinedUnmarkedOperatorQiskit}
\end{equation}
%TODO: Decidir se menciono as equacoes do shift e da coin.
where $S$ is a shift operator, defined in equation \ref{eq:coinedShiftOperator} as 
\begin{equation}
          S = \ket{0}\bra{0} \otimes \sum_{x=-\infty}^{x=\infty} \ket{x+1}\bra{x} + \ket{1}\bra{1}\otimes \sum_{x=-\infty}^{x=\infty} \ket{x-1}\bra{x},
	  \label{eq:shiftMatrixQiskit}
\end{equation} 
that increments or decrements the position of the walker according to the coin operator $C$.\par
%TODO: Reescrever prox paragrafo.
Previously, this system was simulated in Python by implenting it's equations. Now, the focus is to study a quantum circuit based on the work presented by \cite{douglaswang07}. This approach relies on multi-controlled CNOT gates in order to shift the state of the walker by $+1$ or $-1$, each with a probability associated with the chosen coin, as can be seen in figure \ref{fig:douglasWangShift}. 

%TODO: Decidir se ficam 3 pontos verticais.
%TODO: Substituir todas as minipages por subfigures.
\begin{figure}[!h]
  \centering
  \begin{subfigure}[t]{.4\textwidth}
    \centering
    \includegraphics[width=\linewidth]{img/QCircuit/CoinedQuantumWalk/DouglasWangIncrement.png}
    \caption{Increment}
  \end{subfigure}
  %\hfill
  \begin{subfigure}[t]{.4\textwidth}
    \centering
    \includegraphics[width=\linewidth]{img/QCircuit/CoinedQuantumWalk/DouglasWangDecrement.png}
    \caption{Decrement}
  \end{subfigure}
  \caption{Shift operators.}
  \label{fig:douglasWangShift}
\end{figure}

%TODO: Melhorar o paragrafo seguinte
The generalized CNOT gates act on the node states as a cyclic permutator, where each node is mapped to an adjacent state. This can be seen as the walker moving left or right, in the uni-dimensional graph example.\par

%TODO: Descobrir como separar a moeda do resto do circuito. Adicionar node e subnode?
%TODO: Fazer um circuito para N=3 ou N=4 e por os resultados da simulacao com esse numero. 
%TODO: Resize da figura?
\begin{figure}[!h]
	\[ \Qcircuit @C=1.5em @R=1.3em {  & & &  \mbox{Repeat for steps.} & & &  \\ 
	               &  {/^{\otimes n}}\qw  &\qw           & \gate{\mbox{INCR}}     &  \gate{\mbox{DECR}}    & \qw \\
				   & \qw                             & \gate{\mbox{C}}                & \ctrlo{-1}           & \ctrl{-1}                   & \qw \gategroup{2}{3}{3}{5}{.8em}{--}      
		          } \]
	\centering
	\caption{Coined quantum walk circuit}
	\label{fig:coinedCircuit}
\end{figure}

The coin operator will simply be a Hadamard gate acting on a single qubit. For a graph with $16$ nodes, for example, $4$ qubits are required to encode each node and an extra qubit for the coin. The circuit will then be as shown in figure \ref{fig:coinedCircuit}. Note that this circuit limits the number of graph nodes to powers of $2$, and an arbitrary implementation of $2^n$ nodes requires $n+1$ qubits.
%TODO: Decidir como falar sobre como implementar nos que nao sao potencias de 2. Deverei mencionar a figura do artigo do doulgas wang?
%TODO: Descobrir uma source diferente para o gray code ordering. Nielsen chuang.
However, it is possible to have any number of nodes, given that the proper correction is made as can be seen in \cite{douglaswang07}. The method used was \textit{Gray Code Ordering} proposed by \cite{alexslepoy06}, whereby a certain arrangement of CNOT gates results in control states only differing by a single bit.\par

\begin{figure}[!h]
	\centering
	\includegraphics[scale=0.32]{img/Qiskit/CoinedQuantumWalk/Circuits/circCoinedQW_N3_S3.png}
	\caption{Temp} 
	\label{fig:coinedQWCircuitQistkit}
\end{figure}
%TODO: Fazer um bloco para condicao init.

This circuit was implemented in Qiskit, as can be seen in figure \ref{fig:coinedQWCircuitQistkit}. In this example, the increment and decrement sequence was applied three times on a graph of size $2^3 =8$ nodes. The starting position of the walker was set to $\Psi(0) = \ket{4}$, and the Hadamard coin was used. The first block after the barrier is the sequence of operations that will increment the state of the walker, as is shown in figure \ref{fig:incrCircuitQistkit}. The generalized NOT gates are implemented using Qiskit's \textit{mcx} function, which implements the decomposition present in lemma 7.5 of \cite{barenco95} to create CNOT gates up to four controls, and the aforementioned Gray Code for more generalized instances of the gate.
%TODO: Decidir se crio um anexo a explicar o lema 7.5 do barenco.
%The circuit is simply the CNOT decomposition of figure \ref{fig:generalDecompCircuit} applied to the increment circuit of figure \ref{fig:coinedIncrement} for the $N=4$, qubit case.
\begin{figure}[!h]
	\centering
	\includegraphics[scale=0.32]{img/Qiskit/CoinedQuantumWalk/Circuits/circIncr_N3_S3.png}
	\caption{Temp} 
	\label{fig:incrCircuitQistkit}
\end{figure}
The next set of operations perform the decrement of the state of the walker, which is just an increment block with it's controls negated as is shown in figure \ref{fig:decrCircuitQistkit}. The rest of the circuit is just the repetition of these operations as a function of the number of steps required.

\begin{figure}[!h]
	\centering
	\includegraphics[scale=0.32]{img/Qiskit/CoinedQuantumWalk/Circuits/circDecr_N3_S3.png}
	\caption{Temp} 
	\label{fig:decrCircuitQistkit}
\end{figure}
Lastly, the circuit is measured and the results can be seen in figure \ref{fig:coinedQWQiskitDist}. These results can be verified by calculating the time evolution of the wave function associated with the system
\begin{gather}
	\ket{\Psi(0)} = \ket{4}\\
	\ket{\Psi(1)} = \frac{\ket{0}\ket{x=3}+\ket{1}\ket{x=5}}{\sqrt{2}} \\
	\ket{\Psi(2)} = \frac{\ket{0}\ket{x=2}+\ket{1}\ket{x=4}+\ket{0}\ket{x=4}-\ket{1}\ket{x=6}}{2}\\
	\ket{\Psi(3)} = \frac{\ket{1}\ket{x=1}-\ket{0}\ket{x=3}+2(\ket{0}+\ket{1})\ket{x=5}+\ket{0}\ket{x=7}}{2\sqrt{2}}.
\end{gather}
Taking the modulus squared of the amplitudes associated with the states, confirms that the probability distribution associated with the \textit{Qasm} simulator, presented in figure \ref{fig:coinedQWQiskitDist} as the blue bar plot, is correct. 
\begin{figure}[!h]
	\centering
	\includegraphics[scale=0.40]{img/Qiskit/CoinedQuantumWalk/CoinedQW_N3_S0123.png}
	\caption{Temp} 
	\label{fig:coinedQWQiskitDist}
\end{figure}
%TODO: Mencionar o numero de CNOTS?
However, the results obtained from running in IBM's \textit{Toronto} backend are not satisfactory. This is because of the size of the circuit, the coined quantum walk model besides requiring an extra qubit for the coin, also needs a very large number of CNOT gates, which have an average associated error of  $1.284e-2$ in this specific backend. For a better analysis of the error, one can calculate the fidelity between the ideal distribution $p(x)$ and the experimental $q(x)$ using the formula
%TODO: Como so estou a calcular para 1 experiencia, nao preciso do primeiro somatorio?
%TODO: p(x) ou p(x,t) ?
%TODO: Intervalo de confiança?
\begin{equation}
    F(p,q) = \sum_{x=0}^{N-1} \sqrt{p(x)q(x)}
    \label{eq:avgFid}
\end{equation}
where $x$ is the vertex. For $0$ steps, the obtained fidelity is approximately $0.91$ which is very high, because the circuit is simply the initial condition, and when the circuit is increased to $1$ step the fidelity lowers to $0.49$, due to the increase of CNOTS. However, for $2$ and $3$ steps, the fidelity increases again to approximately $0.63$ due to the fact that the probability distribution in the simulator becomes more spread out with the increase of steps, meaning that the random uniform distribution obtained from the noisy experiments ran in IBM's backend will be closer to them than a highly localized distribution like in the case of $1$ step.\par
In order to curb the effects introduced by noise, the next chapter will present the circuit for the alternative discrete model, the staggered quantum walk. Besides not needing an extra qubit for the coin, this model also does not require as many CNOT gates for each iteration.

\end{document}

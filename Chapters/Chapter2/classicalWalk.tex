\documentclass[../../dissertation.tex]{subfiles}
\begin{document}
The term \textit{random walk}, firstly introduced by \cite{kpearson1905}, is classically defined as a stochastic process that models the path a walker would take through a mathematical space, where each step made by the walker is random. This can be used to model systems such as a molecule displaying Brownian motion in a fluid, or even fluctuating stock prices as can be seen in \cite{sottinen2001}.\par 
The simplest instance of this walk is on a discretely numbered line, whose mathmatical space is composed of integer numbers. Here, the walker can only advance with equal probability in one of two directions, depending on the outcome of a random event such as tossing a coin. This was coded in Python, and the result of iterating the walk several times is a binomial distribution centered around the starting position.\par
\begin{figure}[!h]
	%TODO: Decidir se faco outra fig para a evolucao do desvio padrao ou algo do genero. 
	\centering
	\includegraphics[scale=0.40]{img/ClassicalWalk/MultClassicalWalk72180450}
	\caption{Classical Walk Temp.} 
	\label{fig:MultClassicalWalk72180450}
\end{figure}
%TODO: Decidir se menciono a equacao da probabilidade para explicar o facto de que a distribuicao e zero em t+n=impar, que e revelevante no coined quantum walk. Isto implica talvez ter que mencionar a probabilidade para as outras caminhadas, que nao sei se vale a pena. 
The number of steps (iterations) directly affects how far the walker can reach, as can be seen in figure \ref{fig:MultClassicalWalk72180450}. As the number of steps increases, the height of each curve at the starting position decreases and the width of the curves increases. This relationship can be captured by the \textit{position standard deviation}, and \cite{REN1} shows that the standard deviation is
\begin{equation}
	\sigma(t) = \sqrt{t}.
	\label{eq:classicalWalkDeviation}
\end{equation}
%TODO: Talvez escrever mais um bocado nesta proxima linha.
In other words, equation \ref{eq:classicalWalkDeviation} represents the rate at which a walker moves away from the origin.\par
%TODO: Melhorar este fecho. Nao sei quao relevante e o que inclui.
Note that this algorithm can be abstracted to graphs of higher dimensions. For example, in a two dimensional lattice, a walker would be transversing a plane with integer coordinatesi, choosing one of four directions in every intersection. Notably, \cite{polya1921} proved that a walker in a two dimensional lattice will almost surely return to the origin at some point. However, the probability of returning to the origin decreases as the number of dimensions increases, as shown by \cite{montrol1956} and \cite{finch2003}.\par
It is worth noting that a random walk, over a graph whose nodes are weighed and directed, is analagous to a \textit{discrete-time Markov chain}\footnote{A Markov chain can be described as a sequence of stochastic events where the the probability of each event depends only on the state of the previous event.}.\par
%TODO: Pensar numa melhor descricao para os proximos capitulos.
The following sections will be used to describe various models of a quantum counterpart of the classical random walk. 
%It can be shown that the  standard deviation is $\sqrt{t}$, where \textit{t} is the number of time steps. The one dimensional walk can be abstracted to a graph of any dimension. This is known as the classical random walk and when the graph has its nodes weighted and it's edges directed, it is analogous to a discrete-time Markov chain 
%TODO: \textcolor{red}{Você precisa esclarecer desde o início que está falando de uma caminhada clássica aqui, referenciar o livro do renato portugal, por exemplo, sobre o desvio padrão. Se quiser, pode fazer uma simulação clássica ou até mesmo abrir uma seção antes só para isso. Ah, precisa dizer que é uma linha infinita.}\par    
\end{document}

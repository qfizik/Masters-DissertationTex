\documentclass[../../dissertation.tex]{subfiles}
\begin{document}

%TODO: Como ainda nao fiz o chap2 este paragrafo esta incompleto.
The motivation behind the contents in this dissertation is to create an
expanded overview on  the topic of quantum random walks. To better
contextualize the rest of this thesis, chapter \ref{chap:QuantumComp} presents
the basic concepts and mathematical tools needed to study quantum walks. \par

Chapter \ref{chap:QuantumWalks} consists of the study of three major quantum
random walk models, more specifically the discrete-time coined quantum walk,
the continuous-time quantum walk and the staggered quantum walk. For each of
these models, this work presents the theoretical framework as well as Python
implementations. In these simulations, the dynamics of the walks are analised
by changing various parameters, plotting the resulting probability
distributions and seeing how these parameters alter the shape, propagation and
other features of the quantum random walks.\par

Chapter \ref{chap:searchingProblems} follows this approach, but now the
structure where these walks take place is a complete graph and the goal is to
find a marked element. For this purpose, the section about Grover's algorithm
is used to introduce the notion of quantum searching problems. The following
sections show how to change the various models of quantum walks to accomodate
an oracle, thus performing an element search in time similar to Grover's
algorithm. \par

Finally, chapter \ref{chap:qiskitImplementation} is dedicated to constructing
circuits for the models previously defined, using IBM's software
\textit{Qiskit} and their hardware. The first three sections are used to create
circuits for the dynamics of the walks. The biggest contribution being the
circulant graph approach to building diagonal operators for the continuous-time
quantum walk, which can be easily translated to Qiskit circuits. Although work
by \cite{qiang2016} pioneered this approach, the work presented in this thesis
aims to give a clear description on how to build these circuits in Qiskit and
an original analysis on how the approximate quantum Fourier transform affects
the accuracy of the results and the number of operations needed to perform the
quantum walk.
%TODO: Falar sobre NISQ. Artigo preskill
%TODO: Ficou um bocado incompleta esta parte.
The last section shows how introduce an oracle to the various circuits in order
to perform a searching problem. For the continuous-time case, circulant graphs
are again used in an original implementation of the searching problem making
use of diagonal operators and the Suzuki-Trotter expansion.

%BRUNO: eu colocaria os trabalhos aqui (artigo + pôster), ou na referências e citaria no texto

\end{document}


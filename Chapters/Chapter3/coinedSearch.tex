\documentclass[../../dissertation.tex]{subfiles}
\begin{document}
In classical computation, a \textit{spatial search problem} focuses on finding
marked points in a finite region of space. Defining this region with graphs is
fairly straightforward, the vertices of the graph are the search space, and the
edges define what transitions are possible through the search space. As was
previously mentioned in section \ref{sec:GrovSearchSimul}, exhaustively
searching through an unstructured space, by means of a classical random walk
for example, would mean that in the worst case, one would have to take as many
steps to find the marked points as there are vertices in the graph. Quantum
computing provides an alternative to this complexity through Grover's
algorithm, and applying some of his ideas to the coined quantum walk not only
allows a quantum counterpart to the random walk search, but also further
insight into the algorithm itself.\par

%TODO:\textcolor{red}{acho que você precisa delimitar o que fará ao longo desta seção, dizer que tratará os três modelos e a busca}
Following \cite{REN1}'s definition, a good first step is to borrow the
diffusion from Grover's algorithm and invert the sign of the state
corresponding to the marked vertex while leaving unmarked vertices unchanged.
This is done through the following operator 
%TODO:\textcolor{red}{eu trocaria a notação de $\mOathcal{F}$ por $\mathcal{O}$ e dizer que é um oráculo}
\begin{equation}
	\mathcal{O} = I - 2 \sum_{x\in M} \ket{x}\bra{x}
\end{equation}
where M is the set of marked vertices and $\mathcal{O}$ is an analogue to
Grover's oracle. For one marked vertex, this oracle can be written as 
%TODO:\textcolor{red}{você pode até dizer que $M=\{0\}$}
\begin{equation}
	\mathcal{O} = I - 2 \ket{0}\bra{0}
\end{equation}
Notice that there is no loss of generality by choosing the marked vertex as
$0$, since the labelling of the vertices is arbitrary.\par 

The next step is to combine the evolution operator from the coined quantum walk
model with the oracle
\begin{equation}
	U'= U\mathcal{O}
	\label{eq:43}
\end{equation}
Similarly to the simple coined case, the walker starts at $\ket{\Psi(0)}$ and
evolves following the rules of an unitary operator $U$ followed by the sign
inversion of marked vertices. The walker's state after an arbitrary number of
steps will be
\begin{equation}
	\Psi(t) = (U')^t\ket{\Psi(0)}.
	\label{eq:sysStateSearch}
\end{equation}\par

For a better understanding of the search problem in the coined quantum walk
model, consider a graph where all the vertices are connected and each vertex
has a loop that allows transitions to itself.  The next step is to label the
edges using notation $\{(v,v'), v \geqslant 0 \land v' \leqslant N-1\}$ where
$N$ is the total number of vertices and $(v,v')$ are the position and coin
value, respectively.  The shift operator, now called \textit{flip-flop} shift
operator, is
\begin{equation}
	S\ket{v1}\ket{v2} = \ket{v2}\ket{v1}.
	\label{eq:chap3FlipFlop}
\end{equation}\par
The coin operator is defined as
\begin{equation}
	C = I_N \otimes G
\end{equation}
where 
%TODO:\textcolor{red}{prefiro s a D}
\begin{equation}
	G = 2\ket{D}\bra{D} - I
\end{equation}
is the Grover coin, with $\ket{D}$ being the diagonal state of the coin space.\par

Marking an element in a complete graph is done through the following oracle
\begin{equation}
	\mathcal{O'} =\mathcal{O}\otimes I = (I_N - 2\ket{0}\bra{0})\otimes I_N = I_{N^2} - 2 \sum_v \ket{0}\ket{v}\bra{0}\bra{v} ,
\end{equation}
that can be seen as an operator that marks all edges leaving $0$. Recalling
\ref{eq:43}, now that all the operators are defined, the modified evolution
operator can then be written as
\begin{equation}
	U' = S(I \otimes G)\mathcal{O'} = S(I \otimes G)\mathcal{O} \otimes I = S (\mathcal{O} \otimes G),\label{eq:modifiedEvoCoined}
\end{equation}
and the state of the system will evolve according to equation
\ref{eq:sysStateSearch}.\par 

As was shown in \cite{REN1}, maximum probability of the marked vertex is
achieved after $\frac{\pi}{2}\sqrt{N}$ steps. Figure \ref{fig:coinedSearch} is
the result of coding and plotting the evolution of this probability
distribution, for graphs of varying sizes. It shows that the probability is
close to one at \textit{approximately} the predicted ideal steps, because of
the discrete nature of the walk. The probability distributions have a
stair-like shape, because transitions in this model only occur on even numbered
time steps, because of how the unmodified evolution
operator was constructed.
\begin{figure}[!h]
	\centering
	\includegraphics[scale=0.40]{img/CoinedQuantumWalk/Search/CoinedSearch163264.png}
	\caption{Probability of one marked element in the coined quantum walk search, as a function of the number of steps, for complete graphs of size $N=16$, $32$ and $64$.}\label{fig:coinedSearch}
\end{figure}\par
The following section is dedicated to the study of the search problem using the
staggered quantum walk model. The algorithm is still discrete, however since it
does not use a coin, its Hilbert space will be much smaller. In the coined
case, due to how the coin and shift operators were defined, for every qubit
that represents the space of the walker, another qubit will be needed for the
coin. This means that for a $N$ qubit walk, $2N$ qubits will be needed.
Therefore, the staggered quantum walk will be better suited for running in a
NISQ computer.

\end{document}

\documentclass[../../dissertation.tex]{subfiles}
\begin{document}

Following the previous chapters structure, here the study of the search problem
using quantum walks will be presented. Section \ref{sec:GrovSearchSimul}
introduces the basics of the search problem by presenting the theoretical
framework of Grover's algorithm, followed by a complexity analysis together
with Python plots for a better ilustration. Different numbers of marked
elements will be shown and, by the end of the section, it should be clear that
Grover's algorithm is optimal for searching, as was shown by \cite{zalka1999}.\par

Subsequent sections are dedicated for the quantum walk instance of this
problem. In section \ref{sec:CoinedSearchSimul}, the coined quantum
walk is defined for the search problem, which implies the introduction of an
oracle. Here, instead of a line, a complete graph is used, which will
increase the space of the search to $2N$, due to the connected nature of this
graph and the need of a coin. Section \ref{sec:StagSearchSimul} presents the
staggered quantum walk version of the search problem. Again, the notion of
cliques and tesselations is used instead of a coin, and the complete graph is
again considered. The oracle is again defined for this walk, which makes it
quite similar to Grover's algorithm. However, since it is possible to alter the
parameter $\theta$ and the structure over which the search is performed, this
algorithm is known to be a more general case of Grovers. Like the coined
quantum walk, the staggered model is discrete in time but, since a coin is not used,
the space associated with it scales only with the size of the graph, meaning
its implementation on a NISQ computer will be more feasible. Finally, section
\ref{ec:ContSearchSimul} closes this chapter with the search instance of the
continuous-time quantum walk. Similarly to the staggered quantum walk, both the
structure where the search is performed and parameter$\gamma$ can be
controlled. However, since time is not discrete in this instance, the
probability distribution associated with this walk will be slightly different,
and it will later be seen that this results in a circuit that does not scale up
with time.  


\end{document}
